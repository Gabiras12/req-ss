\chapter[Elicitação de Requisitos]{Elicitação de Requisitos}\label{cap1}
  A elicitação de requisitos é uma atividade fundamental da Engenharia de Requisitos,
e para a contemplação dessa atividade, foram desenvolvidas algunas técnicas de elcitação, que serão descritas na
próxima sessão.

\section{Técnicas de Elicitação de Requisitos}

  Existem várias técnicas de elicitação, cada uma delas possui suas características próprios,
alem de suas vantagens e desvantagens, cabe a equipe a responsabilidade de selecionar a técnica ou técnicas
que melhor se adequam ao seu contexto para que seja obtido requisitos concisos, portando a seguir serão
apresentadas as técnicas que foram definidas pelo grupo como as mais adequadas a realidade do projeto.
\subsection{Workshop}
  Workshop é uma técnica de elicitação em grupo feita em uma reunião estruturada,
o grupo consiste de uma seleção de stakeholders, representando a organização e o contexto em que o
sistema será usado, de uma equipe de analistas e um participante neutro, que é o responsável por conduzir e
promover a discursão entre os mediadores, um facilitador.

  A intenção é promover a interação, o trabalho em equipe, ao contrário de reuniões por exemplo.
Uma técnica utilizada no workshop é o brainstorming, que seja explicado na próxima sessão
No fim do processo são produzidos documentos que refletem os requisitos e decisões acertadas sobre
o sistema a ser desenvolvido.
\subsection{Brainstorming}
  A principal ideia da técnica de brainstorming é permitir a geração de ideias,
que são exploradas durante a técnica. Para a realização de uma sessão de brainstorming é
necessário que se solicite participantes que possam contribuir diretamente com a sessão,
ou seja, pessoas bem informadas e vindas de diferentes frutos garantira uma boa representação.
Na execução da sessão, os convidados são convidar, um por vez, a dar uma única ideia sobre o tópico em discursão,
caso haja algum problema com algum participante, esse passa a vez e espera a próxima rodada.
Os participantes também são encorajados a combinar ou enriquecer as ideias dos colegas, por isso é importante
que todas ideias estejam acessíveis a os participantes.

  Nessa técnica as ideias são encorajadas, podendo parecer as vezes não convencionais,
por outro lado elas estimulam os participantes, o que pode culminar em soluções criativas para o problema.
A intenção é que se tenha uma quantidade grade de ideias, pois quanto maior o número de ideias,
maior é a chance da obtenção de boas ideias.

  A última fase desta técnica, consistem em analisar as ideias,
revisando uma de cada vez e as consideradas valiosas são mantidas e classificadas em ordem de prioridade.

\subsection{Entrevista}
  Apesar de ser, dentre as técnicas mais tradicionais, a mais simples de executar,
a entrevista pode produzir bons resultados na fase inicial de obtenção de informações.
O sucesso nessa técnica é obtido desde que se faça um planejamento da entrevista,
para que haja uma linearidade e evitando a dispersão do assunto, além da falta de planejamento
possibilitar uma extensão muita grande na conversa, tornando-a cansativa e prejudicado a qualidade
dos dados coletados. É importante também ressaltar que o entrevistador precisa dar liberdade para que o
entrevistado exponha suas ideias.

  Desenvolver um plano geral de entrevistas, planejar a entrevista para fazer uso eficiente do tempo,
utilizar ferramentas automatizadas que sejam adequadas, tentar descobrir que informação o usuário está mais
interessado e usar um estilo adequado ao entrevistar, são diretrizes que pode contribuir para o sucesso de
uma entrevista.

  Primeiramente, é preciso que se tenha uma contextualização, para que se tenha um maior proveito na elaboração
dos assuntos a serem discutidos na entrevista. Essa contextualização é obtida através de formulários,
relatórios, documentos entre outros. Com isso pode-se delimitar um escopo, que deve ser relativamente limitado,
para que a entrevista não se estenda por mais de 1 hora, visto que há uma dificuldade de concentração em reuniões
muito longas. Além disso, deve-se evitar o excesso de termos técnicos e não conduzir a entrevista em uma tentativa
de persuasão, simplicidade e clareza é o melhor caminho. É importante ressaltar, que caso o contexto seja extenso,
a ideia é dividir e focalizar em apenas uma parte do sistema de cada vez.

  Após a entrevista é necessário que se haja uma validação do que foi documento, afim de ser acordado se a
informação adquirida condiz com as necessidades do cliente, isso pode ser feito do procedimento de confirmação,
onde o entrevistado deve dizer ao usuário o que acha que ouviu ele dizer, utilizando suas próprias palavras e
solicitar a confirmação por parte do entrevistado. É importante reconhecer que o entrevistado é o
perito no assunto e fornecerá as informações necessárias ao sistema.
