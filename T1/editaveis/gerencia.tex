\chapter[Gerenciamento de Requisitos]{Gerenciamento de Requisitos}\label{cap1}
\section{Rastreabilidade de Requisitos}
  Rastreabilidade é a propriedade de uma especificação de requisitos que reflete a
facilidade de encontrar os requisitos relacionados (SOMMERVILLE, 2007, p. 108).

  Assim a rastreabilidade se torna vital a qualquer processo de software, pois a partir de
uma rastreabilidade bi-direcional descrita por (DAVIS, 1993) sendo a capacidade de rastrear um
requisito até seus refinamentos é definida como rastrear para frente (Forwards), e a de
rastrear um refinamento até sua origem é definida como rastrear para trás (Backwards) é possível
rastrear objetos de origem/destino a partir de qualquer ponto, refletindo a qualidade que (PRIBERAM, 2013)
diz: Denota a qualidade do que é rastreável. A capacidade para acompanhar o percurso de um produto, ou de
conhecer o seu processo de produção, manipulação, transformação, embalagem ou expedição.

  Assim, neste projeto tomamos como estratégia a rastreabilidade bi-direcional dos requisitos, garantindo
   que a partir de um tema de investimento seja possível rastrear todas as suas derivações,
como ilustrado no esquema abaixo:

\begin{figure}[H]
    \centering
	\includegraphics[keepaspectratio=true,scale=0.5]{figuras/traceability.eps}
    \caption{Mapa Mental - Ilustração de rastreabilidade de requisitos}
    \label{fig:processo}
\end{figure}

\section{Atributos de Requisito}

  Atributos são uma fonte muito importante de informação sobre requisitos é a partir deles que se sabe a
origem, a sua importância relativa e a data em que foi criado. Se criados apropriadamente,
eles podem fornecer informações significantes sobre o estado do sistema.
Assim como é possível executar consultas para encontrar todos os requisitos
concluídos ou de alta prioridade.
  Dentre os tipos de atributos existentes, iremos utilizar:
\subsection{Benefício}
  Indica o grau de benefício ou prioridade dos requisitos em relação às
expectativas dos Fornecedores de Requisitos.

\begin{table}[H]
  \centering
    \begin{tabular}{| m{5em} | m{10cm} |}
      \hline
      Crítico     & Requisitos essenciais cujo fracasso em sua implementação significa que o sistema não irá atender as necessidades dos interessados. Imprescindível que seja atendido pelo sistema, condição fundamental para o sucesso do projeto   \\ \hline
      Importante & Requisitos importantes para a eficácia ou eficiência do sistema. Sua não implementação afeta a satisfação do usuário e/ou o valor agregado do produto e o não atendimento não determina o fracasso do projeto. \\ \hline
      Útil & Requisitos úteis, porém menos críticos, sendo usados menos freqüentemente. Não possui muito significado para a satisfação do usuário e pode deixar de ser atendida.   \\ \hline
    \end{tabular}
    \caption{Atributo: Benefício }
    \label{tabela:atributo_beneficio}
\end{table}

\subsection{Estabilidade}
  Indica o grau de maturidade e confiabilidade em relação ao entendimento e
comprometimento de um requisito entre os envolvidos do projeto.

\begin{table}[H]
  \centering
    \begin{tabular}{| m{5em} | m{10cm} |}
      \hline
      Alta     & Deve indicar requisitos que possuem alto grau de estabilidade. O entendimento pela Equipe de Projeto e pelos Fornecedores de Requisitos é evidente e a probabilidade de ocorrência de mudanças é baixa.   \\ \hline
      Média & Requisitos cuja   probabilidade de ocorrência de mudanças é considerável. Alguns fatores que determinam esse valor são: requisitos ainda não entendidos completamente pela equipe de projeto ou com algumas pendências de definições por parte do Fornecedor do Requisito. \\ \hline
      Baixa & Requisitos cuja mudança é certa, devido à baixa maturidade do mesmo. Requisitos altamente complexos, requisitos com muitas pendências de definição, requisitos com   histórico de mudanças elevadas e requisitos com influências externas (Leis, outros sistemas) são alguns fatores que influenciam para esse valor.   \\ \hline
    \end{tabular}
    \caption{Atributo: Estabilidade }
    \label{tabela:atributo_estabilidade}
\end{table}

\subsection{Complexidade}
  Indica o nível de esforço necessário ou o quão difícil é a implementação do requisito.
\begin{table}[H]
  \centering
    \begin{tabular}{| m{5em} | m{10cm} |}
      \hline
      Pontos     &  Pontos de de 0 a 100 na escala fibonacci que indicam o nível de esforço  \\ \hline
  \end{tabular}
  \caption{Atributo: Complexidade }
    \label{tabela:atributo_estabilidade}
\end{table}

\subsection{Situação}
   Indica a situação atual de um requisito.
\begin{table}[H]
  \centering
    \begin{tabular}{| m{5em} | m{10cm} |}
      \hline
      Proposto     & Indica requisitos que foram solicitados pelos Fornecedores de Requisitos, mas ainda estão em análise pela Equipe de Projeto ou pelo Cliente.   \\ \hline
      Aprovado & Requisito aprovado e incorporado ao escopo do sistema. \\ \hline
      Cancelado & Requisito que foi cancelado. Para ser cancelado o requisito pode estar Aprovado, e neste caso será retirado do escopo do sistema, ou pode estar simplesmente Proposto.   \\ \hline
    \end{tabular}
    \caption{Atributo: Situação }
    \label{tabela:atributo_estabilidade}
\end{table}

\subsection{Responsável}
  Indica o nome do responsável na Equipe de Projeto pelo requisito.
\begin{table}[H]
  \centering
    \begin{tabular}{| m{5em} | m{10cm} |}
      \hline
      Membro da Equipe     & Estabelece o membro da equipe responsável no momento pelo requisito. Pode ser alterado durante o tempo.   \\ \hline
    \end{tabular}
    \caption{Atributo: Responsável }
    \label{tabela:atributo_estabilidade}
\end{table}

\subsection{Observações}
  Atributo livre para registro de observações como pendências ou problemas que atualmente está ocorrendo com o requisito.
\begin{table}[H]
  \centering
    \begin{tabular}{| m{5em} | m{10cm} |}
      \hline
      Texto Livre     & Importante recurso   para que o Analista de Requisitos ou o Coordenador de Projetos registre   pendências encontradas no desenvolvimento do requisito.   \\ \hline
    \end{tabular}
    \caption{Atributo: Observações }
    \label{tabela:atributo_estabilidade}
\end{table}
