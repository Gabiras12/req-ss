\chapter[Considerações Finais]{Considerações Finais}\label{cap1}
  A partir da elaboração desse documento fica nítido a importância da Engenharia de requisitos dentro de um projeto, visto que esse é o primeiro passo no desenvolvimento do sistema possuindo um impacto significativo sobre o sucesso do projeto. Ele delimita o escopo do projeto e é a base para todas as outras etapas do projeto. Erros são evitados, qualidade garantida e  consequentemente custos são reduzidos.  

  Identificar e aplicar a técnica mais adequada para a aplicacao de um processo bem definido de engenharia de requisitos depende diretamente da equipe, do time e do produto ou serviço a ser oferecido. Cabe ao engenheiro de software adequar ao processo de desenvolvimento de requisitos, as ferramentas e técnicas utilizadas, objetivando-se obter a melhor percepcao do problema possível no contexto do cliente e das prováveis solucoes.

  Além de delimitar o escopo, como visto nesse documento, é necessário o estabelecimento de processo de obtenção, usos de técnicas de elicitação, estratégias de rastreabilidade, para que seja eliminado os problemas relacionados a obtenção de requisito, alem de obter o controle deles por todo o projeto, garantindo que o produto de software realmente supra as necessidades do cliente.
