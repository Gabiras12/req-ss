\chapter[Processo de Engenharia de Requisitos]{Processo de Engenharia de Requisitos}\label{cap4}

Esse processo foi desenhado utilizando ferramenta ``\textit{BIZAGI modeler}''. Houveram 5 evoluções,
presentes no apêndice A, a partir do primeiro esboço até chegar a versão apresentada abaixo.


Como foi descrito no capítulo 3 o processo criado foi estruturado baseando-seno
SAFe mantendo os três níveis estruturais do framework, portifólio, programa e time
que estão detalhados na sessão 4.1.

Alguns dos pontos cruciais do processo criado, foi manter a rastreabilidade dos requisitos,
uma vez que são definidos vários níveis de requisitos e o processo para manter essa
rastreabilidade  no SAFe não está totalmente definido em atividades.

É fundamental destacar que o problema da escolha de um abordagem não esta em considerar
a mutabilidade dos requisitos e sim em como tratar essas mudanças que são inevitáveis.
Pensando nisso foram adicionadas atividades de gerência de mudanças baseadas no RUP,
para sanar um possível risco na comunicação e gerência do projeto devido a não experiência
de trabalho entre os membros do grupo. Esse processo ocorre de forma eventual e paralela
e será detalhado na sessão 4.2.

\section{Scaled Agile Framework}

Nessa sessão serão descritos os níveis que compoem o SAFe(Portifólio, Programa e Time).
Sendo mostrado o objetivo cada nível como também as atividades que a compoem.

\subsection{Portifólio}

No processo criado nesse trabalho, esse nível tem como objetivo levantar e estabelecer
uma abstração de alto nível dos requisitos do negócio. Esse levantamento ocorrerá
por meio da efetuação das atividades Entender Contexto do Cliente e Levantar Épicos
e Revisar e Analisar. As técnicas utilizadas para a execução de cada atividade foram
a análise documental, workshops e Brainstorms.(MIGUE,2015)

\subsubsection{Atividades}

\begin{table}[H]
    \begin{tabular}{|c|l|}
    \hline
    ID       & 01   \\ \hline
    Nome     & asas   \\ \hline
    Objetivo & asas   \\ \hline
    Entradas & asas   \\ \hline
    Saídas   & asasas \\ \hline
    \end{tabular}
\end{table}
