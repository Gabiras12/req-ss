\chapter[Introdução]{Introdução}\label{cap1}


Este relatório contempla um planejamento da gerência dos requisitos de um produto de software. Esses requisitos serão colhidos com base em um processo organizacional que poderá será analisado e otimizado.

O cliente contemplado no relatório é uma equipe de modelagem de processos  que mapearam o processo organizacional da área de manutenção de software do Ministério das Comunicações. Esse processo que a equipe citada mapeou, sofrerá uma analise pela mesma. Serão identificados pontos do processo que podem ser otimizados, e, alguns deles automatizados com software, aumentando a produtividade da organização. Assim, a equipe de Engenharia de Requisitos irá trabalhar para desenvolver requisitos desses softwares de automatização que poderão surgir.

Para a realização desse desenvolvimento, o relatório detalha na sessão XPTO um processo de ER criado com base nos processos SAFe e RUP, que irá guiar a equipe de ER com diversas atividades a serem realizadas, para se alcançar o objetivo de gerar um produto de software de qualidade e que agregue valor a seu cliente. Essas atividades agregarão as 5 atividades fundamentais da Engenharia de Requisitos: Elicitação, Analise e Negociação, Documentação, Verificação e Validação, Gerência de Requisitos.

Foram definidas técnicas de elicitação, propriedades para a gerência de requisitos e uma ferramenta de gestão de requisitos, para facilitar a implantação do processo.

Em seguida será detalhado um breve resumo do contexto do cliente.

\section[Contexto do Cliente]{Contexto do Cliente}\label{contexto_cliente}
O ministério das comunicações possui um processo de manutenção de software, como ilustrado no apêndice \ref{apendice:current_process_mc}, esse
processo gira em torno de analisar os softwares nos quais foram abertas demandas
e, se possível, corrigir e/ou melhor os softwares. O processo para ser efetivado, eles passam por vários departamentos, essas idas e vindas
em meio a analise tonam particularmente delicado a efetivação da correção ou/e melhoramento do software.

A organização estrutural da instituição é independente e vertical. Sendo bem hierárquica
e rígida a sua estrutura e fazendo todo o processos lento e complicado. Porém, por
outro lado, esse tipo de estrutura torna claras e bem distribuídas as atividades à serem desenvolvidas.
