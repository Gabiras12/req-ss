\chapter[Visão]{Visão}

\section{Introdução}

\subsection{Finalidade}

Esta seção especifica os objetivos desse documento, o problema solucionado e a visão geral da solução proposta.

\subsection{Objetivo}

Esse documento, têm como objetivo geral, fornecer informações referentes a visão geral do Sistema de Gestão de Desenvolvimento do Ministério das Comunicações. Esse sistema trás uma solução às dificuldades no acompanhamento do processo de manutenção dos sistemas internos do ministérios.

\subsection{Visão Geral}

Abaixo encontra-se a visão geral da solução para o problema identificado.

A partir do Tema de Investimento estabelecido, foi indentificado os seguintes Épico:

\begin{table}[H]
	\begin{tabular}{|>{\raggedright}p{4cm}|>{\raggedright}p{10cm}|}
		\hline 
		\multicolumn{2}{|c|}{Caso de Negócio}\tabularnewline
		\hline 
		Para & O gerente e a equipe de desenvolvimento\tabularnewline
		\hline 
		Quem & Gere e desenvolve as demandas aprovadas de manutenção do ministério\tabularnewline
		\hline 
		A & Ferramenta de Gestão do Desenvolvimento\tabularnewline
		\hline 
		É uma & Ferramenta que organiza sistematicamente a demanda a ser desenvolvida\tabularnewline
		\hline 
		Que & Facilita e Agiliza o trabalho da equipe de desenvolvimento\tabularnewline
		\hline 
		Diferente & Da situação atual que tudo é feito manualmente\tabularnewline
		\hline 
		Nossa Soluação & Automatiza a gestão e Centraliza as informações das demandas a serem
		desenvolvidas\tabularnewline
		\hline 
		\multicolumn{2}{|c|}{Escopo}\tabularnewline
		\hline 
		Critérios de Sucesso & Produtividade aumentar em 100\%

		Demandas aprovadas pelo usuário aumentar em 20\%\tabularnewline
		\hline 
		No Escopo & Controle de demandas por parte do Gestor\tabularnewline
		\hline 
		Fora do Escopo & Validação de demandas. Registro de demanda por parte do cliente.\tabularnewline
		\hline 
		Regras de Negócio & \tabularnewline
		\hline 
	\end{tabular}
	\caption{Épico 01}
	\label{Epico_01}
\end{table}

\begin{table}[H]
	\begin{tabular}{|>{\raggedright}p{4cm}|>{\raggedright}p{10cm}|}
		\hline 
		\multicolumn{2}{|c|}{Caso de Negócio}\tabularnewline
		\hline 
		Para & O gerente e a equipe de desenvolvimento\tabularnewline
		\hline 
		Quem & Gere e desenvolve as demandas aprovadas de manutenção do ministério\tabularnewline
		\hline 
		A & Ferramenta de Gestão do Desenvolvimento\tabularnewline
		\hline 
		É uma & Ferramenta que organiza sistematicamente a demanda a ser desenvolvida\tabularnewline
		\hline 
		Que & Facilita e Agiliza o trabalho da equipe de desenvolvimento\tabularnewline
		\hline 
		Diferente & Da situação atual que tudo é feito manualmente\tabularnewline
		\hline 
		Nossa Soluação & Automatiza a gestão e Centraliza as informações das demandas a serem
		desenvolvidas\tabularnewline
		\hline 
		\multicolumn{2}{|c|}{Escopo}\tabularnewline
		\hline 
		Critérios de Sucesso & Produtividade aumentar em 100\%

		Demandas aprovadas pelo usuário aumentar em 20\%\tabularnewline
		\hline 
		No Escopo & Controle de demandas por parte do Gestor\tabularnewline
		\hline 
		Fora do Escopo & Validação de demandas. Registro de demanda por parte do cliente.\tabularnewline
		\hline 
		Regras de Negócio & \tabularnewline
		\hline
	\end{tabular}
	\caption{Épico 02}
	\label{Epico_02}
\end{table}

\section{Descrição dos Usuários}

O produto referente a esse documento de visão tem como usuário os funcionários do Ministério das comunicações a que atuam na area de manutenção do sistema interno do ministério. Na sistuação atual  há dificuldades no acompanhamento do processo de manutenção que, resumidamente, consiste em uma requisição de melhoria, em seguida a implementação e por fim uma avaliação do cliente. 

\subsection{Usuários}

\begin{table}[H]
	\begin{tabular}{|l| p{7cm} | p{6cm} |}
		\hline
		\textbf{Nome} & \textbf{Descrição}\tabularnewline
		\hline
		Cliente & Usuário do sistema responsável pelo pedido da demanda\tabularnewline
		\hline
		Gerente de Manutenção & Responsável pela criação e geração de relatórios da demanda\tabularnewline
		\hline
		Desenvolvedor & Responsável pelo desenvolvimento e atualização do estado da demanda\tabularnewline
		\hline
	\end{tabular}
	\caption{Usuários}
	\label{Usuarios}
\end{table}

\subsection{Ambiente do Usuário}

O acesso dos usuários ao sistema se se dão a partir de diferentes sistemas operacionais, destacando-se o Windows e destribuições Linux, portanto um sistema multiplataforma ou que possa ser utilizado independentemente do sistema operacional se torna obrigatório. As aplicações Web se tornam as melhores soluções para o problema, uma vez que são executadas em diferentes sistemas operacionais sem perda de performance.

\section{Visão Geral do Sistema}

Esse item mostra uma visão bem ampla do sistema, já trazendo soluções e requisitos pertencentes ao mesmo.

\subsection{Perspectiva do Sistema}

Como mencionado anteriormente, o sistema deverá ser desenvolvido em ambiente Web e para a escolha do framework a ser utilizado, três plataformas principais foram analisadas sob diversos aspectos, como mostrados na tabela a seguir:

\begin{table}[H]
	\begin{tabular}{| m{4cm} | m{4cm} | m{4cm} | m{4cm} |}
		\hline
		& \begin{center} \textbf{Ruby on Rails} \end{center} & \begin{center} \textbf{DJANGO} \end{center} & \begin{center} \textbf{NODE.js} \end{center}\tabularnewline
		\hline
		\begin{center} \textbf{Linguagem de Programação} \end{center} & Ruby & Python & JavaScript\tabularnewline
		\hline
		\begin{center} \textbf{Memória Recomendada} \end{center} & 1GB & 128MG & 256MB\tabularnewline
		\hline
	\end{tabular}
	\caption{Plataformas do Sistema}
	\label{Plataformas_do_Sistema}
\end{table}

\subsection{Suposições e Depedências}

Caso ocorram mudanças nos épicos, nas features e/ou em quaisquer itens que foram identificados e estão especificados neste documento, essas mudanças devem ser refletidas no próprio, a fim de evitar a perda da sua integridade.

\section{Features}

A partir dos épicos estabelecidos, pode-se gerar as seguintes Features:

\begin{table}[H]
	\begin{tabular}{|>{\centering}p{4cm}|>{\centering}p{4cm}|>{\centering}p{6cm}|c|}
		\hline 
		Épico & Feature & Descrição & Prioridade\tabularnewline
		\hline 
		\hline 
		E01 Gerenciamnto de Demandas & Manutenção da Demanda &  & Alta\tabularnewline
		\hline 
		E01 Gerenciamento de Demandas & Acompanhamento da Demanda &  & Alta\tabularnewline
		\hline 
		E02 Gerenciamento de Usuários & Manutenção de Usuários &  & Alta\tabularnewline
		\hline 
		Gerenciamento de Usuários & Acesso de Usuários &  & Média\tabularnewline
		\hline 
	\end{tabular}
	\caption{Features}
	\label{Features}
\end{table}

\section{Histórias de Usuário}

A partir dos features estabelecidos, pode-se gerar as seguintes histórias de usuário:

\begin{table}[H]
	\begin{turn}{90}
		\begin{tabular}{|>{\centering}p{4cm}|>{\centering}p{4cm}|>{\centering}p{6cm}|>{\centering}p{6cm}|c|}
			\hline 
			Épico & Feature & História de Usuário & Descrição & Prioridade\tabularnewline
			\hline 
			\hline
			\label{HU01}
			E01 Gerenciamnto de Demandas & F01 Manutenção da Demanda & HU01 Eu como desenvolvedor desejo me vincular à uma demanda para me tornar
			o responsável pelo desenvolvimento da demanda &  & Alta\tabularnewline
			\hline 
			\label{HU02}
			E01 Gerenciamnto de Demandas & F01 Manutenção da Demanda & HU02 Eu como desenvolvedor desejo mover a demanda de estado de desenvolvimento
			para indicar o fluxo de produção da demanda &  & \tabularnewline
			\hline 
			\label{HU03}
			E01 Gerenciamnto de Demandas & F01 Manutenção da Demanda & HU03 Eu como gerente desejo cadastrar demanda para ser desenvolvida &  & \tabularnewline
			\hline 
			E01 Gerenciamnto de Demandas & F01 Manutenção da Demanda & HU04 Eu como gerente desejo editar uma demanda cadastrada para manter a
			consistência &  & \tabularnewline
			\hline 
			E01 Gerenciamento de Demandas & F02 Acompanhamento da Demanda & HU05 Eu como desenvolvedor desejo visualizar a descrição da demanda para
			desenvolver as especificações &  & Alta\tabularnewline
			\hline 
			E01 Gerenciamento de Demandas & F02 Acompanhamento da Demanda & HU06 Eu como gerente desejo gerar relatório sobre as demandas para possível
			análise &  & \tabularnewline
			\hline 
			E01 Gerenciamento de Demandas & F02 Acompanhamento da Demanda & HU07 Eu como usuário desejo visualizar as demandas para acompanhar o fluxo
			de produção &  & \tabularnewline
			\hline 
		\end{tabular}
	\end{turn}
	\caption{Histórias de Usuário}
	\label{Historias1}
\end{table}

\begin{table}[H]
	\begin{turn}{90}
		\begin{tabular}{|>{\centering}p{4cm}|>{\centering}p{4cm}|>{\centering}p{6cm}|>{\centering}p{6cm}|c|}
			\hline 
			Épico & Feature & História de Usuário & Descrição & Prioridade\tabularnewline
			\hline 
			\hline 
			E02 Gerenciamento de Usuários & F03 Manutenção de Usuários & HU08 Eu como gerente desejo associar uma conta a uma categoria de acesso
			para organizar as permissões do sistema &  & Alta\tabularnewline
			\hline 
			E02 Gerenciamento de Usuários & F03 Manutenção de Usuários & HU09 Eu como gerente desejo desativar um usuário para manter apenas usuários
			ativos no sistema &  & \tabularnewline
			\hline 
			E02 Gerenciamento de Usuários & F03 Manutenção de Usuários & HU10 Eu como usuários desejo criar uma conta para poder acessar o sistema &  & \tabularnewline
			\hline 
			E02 Gerenciamento de Usuários & F03 Manutenção de Usuários & HU11 Eu como usuários gostaria de editar minhas informações para mantê-las
			atualizadas &  & \tabularnewline
			\hline 
			Gerenciamento de Usuários & F04 Acesso de Usuários & HU12 Eu como usuários gostaria de acessar o sistema para cooperar com a
			gestão da demanda &  & Média\tabularnewline
			\hline 
			Gerenciamento de Usuários & F04 Acesso de Usuários & HU13 Eu como usuário gostaria de recuperar minha senha para acessar ao
			sistema &  & \tabularnewline
			\hline 
		\end{tabular}
	\end{turn}
	\caption{Histórias de Usuário}
	\label{Historias2}
\end{table}

\section{Regras de Negócio}

Resumo em alto nível de outras características do sistema, tipicamente não funcionais.
Este item traz a especificação dos requisitos que dão suporte para a correta execução dos requisitos funcionais e indicam quais são as limitações do sistema e do seu desenvolvimento.

\subsection{Usabilidade}

O sistema deve ser de fácil utilização, auto-explicativo e deve  ser intuitivo para que o usuário esteja apto a usá-lo no primeiro contato com a plataforma.

\subsection{Confiabilidade}

O sistema deve manter os dados salvos em uma memória para caso ocorra um erro no sistema, os dados digitados não sejam perdidos.
Caso uma falha ocorra, o sistema deve estar apto para utilização em no máximo 1 dia.
O sistema deve ficar disponível pelo menos 21horas por dia (06:00 as 00:00).

\subsection{Desempenho}

O sistema deve levar no máximo 3 segundos para RESPONDER uma requisição.

\subsection{Suportabilidade}

O sistema deve ser Web e deve funcionar no Google Chrome Versão
43.0.2357.81 e no Mozilla Firefox Versão 35.0.1 ou versões superiores a estas.

\subsection{Outros}

\begin{table}[H]
	\begin{tabular}{|c|>{\centering}p{12cm}|c|}
		\hline 
		ID & Regra de Negócio & Vínculo\tabularnewline
		\hline 
		\hline 
		RN01 & O sistema deve ser desenvolvido em Ruby on Rails. & Todo o Sistema\tabularnewline
		\hline 
		RN02 & O sistema deve se comunicar com o banco de dados SQLite para o registro
		das informações. & Todo o Sistema\tabularnewline
		\hline 
		RN03 & O sistema deverá registrar os dados dos clientes de maneira sigilosa
		para garantir a privacidade dos mesmos. & Todo o Sistema\tabularnewline
		\hline 
		RN04 & O sistema deve possuir um controle de acesso. & Todo o Sistema\tabularnewline
		\hline 
		RN05 & O sistema não deve permitir que o usuário esteja logado em duas máquinas
		ao mesmo tempo. & Todo o Sistema\tabularnewline
		\hline 
		RN06 & Um desenvolvedor somente se vincula a uma demanda por vez. & \hyperref[HU01]{HU01}\tabularnewline
		\hline 
		RN07 & Um desenvolvedor sempre pegará a próxima demanda na fila de espera. & \hyperref[HU01]{HU01}\tabularnewline
		\hline 
		RN08 & Novas demandas entram na fila de espera. & \hyperref[HU03]{HU03}\tabularnewline
		\hline 
		RN09 & Demandas marcadas como urgente tem prioridade na fila de espera. & \hyperref[HU03]{HU03}\tabularnewline
		\hline 
		RN10 & Após uma demanda ser adicionada, o sistema envia um id para o email
		do cliente vinculado a demanda. & \hyperref[HU03]{HU03}\tabularnewline
		\hline 
		RN11 & A cada mudança de estado da demanda, o sistema envia uma notificação
		por email para os vinculados à demanda e ao gerente. & \hyperref[HU02]{HU02}\tabularnewline
		\hline 
	\end{tabular}
	\caption{Regras de Negócio}
	\label{Regra_de_Negocio}
\end{table}