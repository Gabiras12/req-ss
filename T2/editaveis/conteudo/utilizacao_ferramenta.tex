\chapter[Utilização da Ferramenta de Gerência de Requisitos]{Gerência de Requisitos}

\section{Nível de Portifólio}

A camada de Portfólio consiste no nivel mais elevado do SAFe, onde programas sao alinhados à
estratégia de negócios da empresa ao longo de linhas da corrente de valor. No presente trabalho,
o nivel de portfólio é destacado a partir das atividades \textbf{Entender o Contexto do Cliente},
\textbf{Levantar Epicos} e \textbf{Selecionar e detalhar épicos} que foram realizadas utilizando as
técnicas de elicitação, workshops e entrevistas. As atas de reunião, jutamente com as entrevistas,
podem ser encontradas no apêndice (COLOCAR O NUMERO DO APENCIE).

\subsection{Requisitos Identificados}

\textbf{Tema de Investimento:} Gestão da Manutenção

Extraido a partir das prioridades da organização, serve para assegurar que o desenvolvimento
do software estará de acordo com a estrategia da organização.

Dado que o gargalo identificado na gestão da manutenção de sistemas do MC e onde o cliente irá investir.
Cada épico foi detalhado utilizando o template de lightweight business case, recomendado pelo SAFe. \cite{scaleP}

\textbf{Épico 01:} EP-01 Gerenciamento de Demandas

Abrange todo o movimento das demandas em torno do fluxo estabelecido, desde sua criacao ate o seu fechamento.

\begin{table}[H]
\centering
\caption{Épico 1}
\label{epic:primeiro}
\begin{tabular}{|c|c|}
\hline
\multicolumn{2}{|m{4cm}|}{\textbf{Caso de Negócio}} \\ \hline
  Para   &   O gerente e a equipe de desenvolvimento        \\ \hline
  Quem         &    Gere e desenvolve as demandas aprovadas de manutenção do ministério        \\ \hline
    A       &     Ferramenta de Gestão do Desenvolvimento      \\ \hline
      É uma     &    Ferramenta que organiza sistematicamente a demanda a ser desenvolvida        \\ \hline
      Que     &     Facilita e Agiliza o trabalho da equipe de desenvolvimento       \\ \hline
      Diferente   &     Da situação atual que tudo é feito manualmente      \\ \hline
      Nossa Soluação  &   Automatiza a gestão e Centraliza as informações das demandas a serem desenvolvidas         \\ \hline
\multicolumn{2}{|l|}{Escopo} \\ \hline
Critérios de Sucesso  &  Produtividade aumentar em 100\% e Demandas aprovadas pelo usuário aumentar em 20\%    \\ \hline
No Escopo & Controle de demandas por parte do Gestor. \\ \hline
Fora do Escopo  & Validação de demandas. e Registro de demanda por parte do cliente.     \\ \hline
\end{tabular}
\end{table}

\textbf{Épico 02:} EP-02 Gerenciamento de Usuários

Interfere no modo de comunicacao entre o usuario e o sistema, considerando diversos aspectos referente ao seu acesso.

\begin{table}[H]
\centering
\caption{Épico 2}
\label{epic:segundo}
\begin{tabular}{|c|c|}
\hline
\multicolumn{2}{|m{4cm}|}{\textbf{Caso de Negócio}} \\ \hline
  Para   &   O gerente e a equipe de desenvolvimento        \\ \hline
  Quem         &    Gere e desenvolve as contas de usuário        \\ \hline
    A       &     Ferramenta de Gestão do Manutenção    \\ \hline
      É uma     &    Ferramenta que organiza sistematicamente a demanda a ser desenvolvida \\ \hline
      Que     &     Facilita e Agiliza o trabalho da equipe de desenvolvimento \\ \hline
      Diferente   &     Da situação atual que tudo é feito manualmente     \\ \hline
      Nossa Soluação  &   Automatiza a gestão e Centraliza as informações das demandas a serem desenvolvidas    \\ \hline
\multicolumn{2}{|l|}{Escopo} \\ \hline
Critérios de Sucesso  &  Produtividade aumentar em 100\% e Demandas aprovadas pelo usuário aumentar em 20\%    \\ \hline
No Escopo & Controle de demandas por parte do Gestor. \\ \hline
Fora do Escopo  & Validação de demandas. e Registro de demanda por parte do cliente.     \\ \hline
\end{tabular}
\end{table}



\section{Nível de Programa}

\section{Nível de Time}

\section{Gerência de Requisitos}

\subsection{Atributos de Requisitos}

\subsection{Rastreabilidade de Requisitos}
