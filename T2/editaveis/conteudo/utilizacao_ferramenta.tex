\chapter[Utilização da Ferramenta de Gerência de Requisitos]{Utilização da Ferramenta de Gerência de Requisitos}

\section{Nível de Portifólio}

A camada de Portfolio consiste no nivel mais elevado do SAFe, onde programas sao alinhados a estrategia de negocios da empresa ao longo de linhas da corrente de valor. No presente trabalho, o nivel de portfolio eh destacado a partir das atividades Entender o Contexto do Cliente e Levantar Epicos, Selecionar e detalhar épicos que foram realizadas utilizando as técnicas de elicitação Workshops e Entrevistas.

\subsection{Requisitos}

\textbf{Tema de Investimento:} Gestãoo da Manutencao

Extraído a partir das prioridades da organização, serve para assegurar que o desenvolvimento do software estará de acordo com a estratégia da organização. Dado que o gargalo identificado foi na gestão da manutenção de sistemas e onde o cliente irá investir.  Cada épico foi detalhado utilizando o template de lightweight business case, recomendado pelo SAFe. 

\section{Nível de Programa}

\section{Nível de Time}

\section{Gerência de Requisitos}

\subsection{Atributos de Requisitos}

\subsection{Rastreabilidade de Requisitos}