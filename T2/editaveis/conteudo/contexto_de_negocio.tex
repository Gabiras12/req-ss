\chapter[Contexto de Negócio]{Contexto de Negócio}

O contexto abordado nesse trabalho, foi o processo de manutenção de software do Ministério das Comunicações, presente
no apêndice (COLOCAR NUMERO DO APENDICE). O processo, tem como pricipal atividade a analise de softwares a partir de
demandas abertas, solicitando correções e/ou melhorias, ou seja, após a abertura de uma demanda de manutenção há
a implementação dessa melhoria quando possível.

A demanda para ser efetivada, primeiramente, passa por diversos departamentos dentro do Ministério das Comunicações.
Um dos problemas enfrentado pelo Ministério é, devido a densa burocracia de aprovações, a ineficiência na validação das
das demandas abertas.

Outro problema identificado é a falta de organização das demandas aprovadas pela equipe de manutenção do Ministério.
Não se sabe o que está sendo feito, quais as próximas atividades, ou quem está resposavel por cada demanda. Assim o
o papel de gerênciamento desse processo se torna intensamente árduo. Fazendo com que o processo, como um todo, se torne
lento e ineficaz.

Após a definição do contexto da atuação da equipe de MPR e Requisitos e os problemas identificados, junto ao cliente,
foi executado o processo definido no Trabalho 1 com o objetivo de identificar areas onde um produto de software poderia
atuar para agilizar e aumentar a produtividade do processo de manutenção do MC, tendo como base o processo proposto por MPR.
Podendo assim melhorar o processo organizacional do Ministério.

Utilizando-se das técnicas da Engenharia de Requisitos em conjunto com as metodologias ágeis, foi visto que após a
validação da demana solicitada está, é encaminhada à equipe de manutenção de software do Ministério. Nesse ponto,
as etapas burocráticas já foram realizadas e a demanda pode ser desenvolvida rápidamente, contudo a não organização das
demandas aprovadas gera um gargalo de atividades desordenadas. Especificando assim, a área onde uma solução de software
auxiliaria no processo organizacional. Uma vez que com esse comportamento da equipe de manutenção, grande parte das demandas abertas só entregues com atraso ou não são
homologadas. Fazendo com que demandas voltem ao estado inicial do processo, gerando re-trabalho.

Outro ponto idetificado foi a falta de informações úteis em um lugar centralizado. Fazendo com que os as pessoas que abriram demans - clientes -
não conseguem saber se a demanda foi aprovado, ou em qual estado está.

\section{Solução}

A solução proposta trata-se de uma ferramenta online que teria como área de abrangência todo o processo da manutenção, desde a inserção da demanda para a manutenção até a homologação, auxiliando a organização das demandas, suprindo pontos como priorização, visualização de status de execução, comunicação. Nas seções seguintes serão apresentados mais detalhes de como foi o processo de identificação e da proposta de solução de software
