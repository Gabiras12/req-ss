\chapter[Contexto de Negócio]{Contexto de Negócio}

O ministério das comunicações possui um processo de manutenção de software, como ilustrado no apêndice XPTO, esse processo gira em torno de analisar os softwares nos quais foram abertas demandas e, se possível, corrigir e/ou melhorar os softwares. 

O processo para ser efetivado, passa por vários departamentos, essas idas e vindas em meio a análise tornam particularmente delicada a efetivação da correção e do melhoramento do software. 

\begin{enumerate}[label=(\roman*)]
\item O problema

A partir do contexto e os problemas da organização identificados pela equipe de MPR juntamente com o cliente, a Equipe de Requisitos começou a executar o processo em busca de encontrar dentro dos problemas o que e onde seria possível investir em uma solução de  software afim de melhorar parte do processo organizacional e assim garantindo valor ao cliente. Para obitenção de posto em práticas, técnicas da Engenharia de Requisitos em conjunto com metodologias ágeis na busca do problema e consequentemente no desmembramento do problema até a solução.

Concluiu-se que ao ser validado o processo é enviado para a equipe de manutenção desenvolver. E é nesta etapa que está localizado alguns dos problemas aos quais uma solução de software auxiliaria na solução, pois muitas demandas atrasam a entrega, não são aprovadas na homologação, gerando re-trabalho na manutenção dos softwares, uma vez que ao ser reprovado a demanda retorna para a fase inicial, outro problema é a falta de controle e visualização efetiva de como está o andamento da demanda dentro do processo, além da falta de feedbacks para o cliente, sendo assim o gargalo principal identificado é a falta de comunicação.

\item Solução

A solução proposta trata-se de uma ferramenta online que teria como área de abrangência todo o processo da manutenção, desde a inserção da demanda para a manutenção até a homologação, auxiliando a organização das demandas, suprindo pontos como prorização de demanda, visualização de status de execução, comunicação, com feedbacks para o cliente e comunicação interna. Nas seções seguintes serão apresentados mais detalhadamente como foi o processo de identificação e da proposta de solução de software

\end{enumerate}