\chapter[Planejamento da Primeira Iteração]{Planejamento da Primeira Iteração}

A seguir pode-se verificar as histórias de usuárias organizadas com suas respectivas features e épicos. Cada história de usuário apresenta:

\begin{itemize}
	\item Título;
	\item Valor de négocio, sendo classificado em 5 nivéis: Deve ter, ótimo, bom, médio e bom ter, seu nível de importância é respectiva a ordem de apresentação, ou seja, deve ter (mais importante) e bom ter (menos importante);
	\item Quem: Indicando o agente;
	\item O que: Indicando o que faz;
	\item Porque: Indicando o objetivo;
	\item Critérios de Aceitação: Indicando formas de usar a funcionalidade implementada em uma história;
	\item Pontos: Indicando o grau de dificuldade de implementação da história.
\end{itemize}

\begin{enumerate}
	\item \textbf{Épico:} Gerenciamento de Demanda
		\begin{enumerate}
			\item \textbf{Feature:} Acompanhamento de Demanda
				\begin{enumerate}
					\item \textbf{História 01}
						\begin{table}[H]
							\begin{tabular}{|p{10cm}|l|}
								\hline 
								Título: Descrever Demanda & Valor de Négocio: Deve Ter\tabularnewline
								\hline 
								\multicolumn{2}{|l|}{Quem: Eu como desenvolvedor}\tabularnewline
								\hline 
								\multicolumn{2}{|l|}{O que: Desejo visualizar a descrição da demanda}\tabularnewline
								\hline 
								\multicolumn{2}{|l|}{Porque: Para desenvolver as especificações}\tabularnewline
								\hline 
								\multicolumn{2}{|l|}{Critérios de Aceitação:}\tabularnewline
								\hline 
								\multicolumn{2}{|r|}{Pontos: 5}\tabularnewline
								\hline 
							\end{tabular}
							\caption{História de Usuário}
							\label{Historia_de_Usuario}
						\end{table}

					\item \textbf{História 02}
						\begin{table}[H]
							\begin{tabular}{|p{10cm}|l|}
								\hline 
								Título: Gerar Relatórios & Valor de Négocio: Bom\tabularnewline
								\hline 
								\multicolumn{2}{|l|}{Quem: Eu como gerente }\tabularnewline
								\hline 
								\multicolumn{2}{|l|}{O que: Desejo gerar relatório sobre as demandas}\tabularnewline
								\hline 
								\multicolumn{2}{|l|}{Porque: Para possível análise}\tabularnewline
								\hline 
								\multicolumn{2}{|l|}{Critérios de Aceitação:}
								\tabularnewline
								\hline 
								\multicolumn{2}{|r|}{Pontos: 8}\tabularnewline
								\hline 
							\end{tabular}
							\caption{História de Usuário}
							\label{Historia_de_Usuario}
						\end{table}

						\item \textbf{História 03}
						\begin{table}[H]
							\begin{tabular}{|p{10cm}|l|}
								\hline 
								Título: Visualizar Demandas & Valor de Négocio: Deve ter\tabularnewline
								\hline 
								\multicolumn{2}{|l|}{Quem: Eu como usuário }\tabularnewline
								\hline 
								\multicolumn{2}{|l|}{O que: Desejo visualizar as demandas}\tabularnewline
								\hline 
								\multicolumn{2}{|l|}{Porque: Para acompanhar o fluxo de produção.}\tabularnewline
								\hline 
								\multicolumn{2}{|l|}{Critérios de Aceitação:}
								\tabularnewline
								\hline 
								\multicolumn{2}{|r|}{Pontos: 13}\tabularnewline
								\hline 
							\end{tabular}
							\caption{História de Usuário}
							\label{Historia_de_Usuario}
						\end{table}
				\end{enumerate}
			\item \textbf{Feature:} Manutenção de Demanda
				\begin{enumerate}
					\item \textbf{História 04}
						\begin{table}[H]
							\begin{tabular}{|p{10cm}|l|}
								\hline 
								Título: Cadastrar Demandas & Valor de Négocio: Deve ter\tabularnewline
								\hline 
								\multicolumn{2}{|l|}{Quem: Eu como gerente}\tabularnewline
								\hline 
								\multicolumn{2}{|l|}{O que: Desejo cadastrar demandas}\tabularnewline
								\hline 
								\multicolumn{2}{|l|}{Porque: Para ser desenvolvida}\tabularnewline
								\hline 
								\multicolumn{2}{|l|}{Critérios de Aceitação:}
								\tabularnewline
								\hline 
								\multicolumn{2}{|r|}{Pontos: 8}\tabularnewline
								\hline 
							\end{tabular}
							\caption{História de Usuário}
							\label{Historia_de_Usuario}
						\end{table}

						\item \textbf{História 04}
						\begin{table}[H]
							\begin{tabular}{|p{10cm}|l|}
								\hline 
								Título: Editar Demanda & Valor de Négocio: Ótimo\tabularnewline
								\hline 
								\multicolumn{2}{|l|}{Quem: Eu como gerente}\tabularnewline
								\hline 
								\multicolumn{2}{|l|}{O que: Desejo editar uma demanda}\tabularnewline
								\hline 
								\multicolumn{2}{|l|}{Porque: Para manter a consistência}\tabularnewline
								\hline 
								\multicolumn{2}{|l|}{Critérios de Aceitação:}
								\tabularnewline
								\hline 
								\multicolumn{2}{|r|}{Pontos: 5}\tabularnewline
								\hline 
							\end{tabular}
							\caption{História de Usuário}
							\label{Historia_de_Usuario}
						\end{table}
				\end{enumerate}
		\end{enumerate}
	\item \textbf{Épico:} Gerenciamento de Usuário
		\begin{enumerate}
			\item \textbf{Feature:} Manutenção de Usuário
				\begin{enumerate}
					\item \textbf{História:} 
				\end{enumerate}
			\item \textbf{Feature:} Acesso de Usuário
				\begin{enumerate}
					\item \textbf{História:} 
				\end{enumerate}
		\end{enumerate}
\end{enumerate}