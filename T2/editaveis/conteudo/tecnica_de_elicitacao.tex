\chapter[Técnicas de Elicitação]{Técnicas de Elicitação}

Uma etapa crítica da engenharia de requisitos é a elicitação, fase complexa da definição de requisitos e de todo processo de desenvolvimento de software, uma vez que é base para todas as etapas posteriores \cite{BELGAMO}.
	
E para que a elicitação seja feita com maestria e sucesso, as técnicas de elicitação são uma base para garantir o entendimento e comunicação entre os interessados, pois  uma das principais dificuldades enfrentadas pelo desenvolvedores de software é a complexidade do domínio do problema \cite{BOOCH} e como as características do problema variam muito, nós precisamos de um repertório de métodos para cada classe de problema \cite{SIDDIQI}. 

As técnicas de elicitação escolhidas para auxiliar na identificação dos requisitos junto aos usuários devem explorar características específicas do problema sendo tratado no levantamento de requisitos. As características dos problemas variam, por isso é necessário um repertório de métodos para cada classe de problemas \cite{BELGAMO}, tendo como escolhidas:

\begin{itemize}
	\item Workshop:
		\begin{itemize}
			\item Foi realizado um workshop no dia 19/10 com o intuito de compreender melhor o contexto do cliente onde a agenda e o relatório se encontram no apendice xpto.
			\item Para o workshop nos orientamos pelo anexo %(UFPR, Unidade 2: Elicitação de Requisitos (Parte c), http://anotacoes-ufpr.googlecode.com/svn/trunk/engenharia_de_requisitos/pdfs/Unidade2-WorkshopsdeRequisitos.pdf )
		\end{itemize}

	\item Brainstorm:
		\begin{itemize}
			\item Foi realizado diversas vezes, uma vez durante o workshop e outras para derivar o tema de investimento e os prováveis épicos, sendo um processo contínuo até chegar nas histórias de usuário, sendo refinado a cada branstorm.
		\end{itemize}

	\item Entrevista:
		\begin{itemize}
			\item Foi realizado durante todo o processo, pois diversas vezes nos encontramos inconsistentes o que gerou constantemente a necessidade de se encontrar com MPR para esclarecer duvidas ou procurar uma solução para elas.
		\end{itemize}
\end{itemize}
