\chapter[Técnicas de Elicitação]{Técnicas de Elicitação}

Uma etapa crítica da engenharia de requisitos é a elicitação, fase complexa da definição de requisitos e de todo processo de desenvolvimento de software, uma vez que é base para todas as etapas posteriores \cite{BELGAMO}.
	
E para que a elicitação seja feita com maestria e sucesso, as técnicas de elicitação são uma base para garantir o entendimento e comunicação entre os interessados, pois  uma das principais dificuldades enfrentadas pelo desenvolvedores de software é a complexidade do domínio do problema \cite{BOOCH} e como as características do problema variam muito, nós precisamos de um repertório de métodos para cada classe de problema \cite{SIDDIQI}. 

As técnicas de elicitação escolhidas para auxiliar na identificação dos requisitos junto aos usuários devem explorar características específicas do problema sendo tratado no levantamento de requisitos. As características dos problemas variam, por isso é necessário um repertório de métodos para cada classe de problemas \cite{BELGAMO}, tendo como escolhidas:

\begin{itemize}
	\item Workshop:
		\begin{itemize}
			\item Foi realizado um workshop no dia 19/10 com o intuito de compreender melhor o contexto do cliente onde a agenda e o relatório se encontram no apendice xpto.
			\item Para o workshop nos orientamos pelo anexo %(UFPR, Unidade 2: Elicitação de Requisitos (Parte c), http://anotacoes-ufpr.googlecode.com/svn/trunk/engenharia_de_requisitos/pdfs/Unidade2-WorkshopsdeRequisitos.pdf )
		\end{itemize}

	\item Brainstorm:
		\begin{itemize}
			\item Foi realizado diversas vezes, uma vez durante o workshop e outras para derivar o tema de investimento e os prováveis épicos, sendo um processo contínuo até chegar nas histórias de usuário, sendo refinado a cada branstorm.
		\end{itemize}

	\item Entrevista:
		\begin{itemize}
			\item Foi realizado durante todo o processo, pois diversas vezes nos encontramos inconsistentes o que gerou constantemente a necessidade de se encontrar com MPR para esclarecer duvidas ou procurar uma solução para elas.
		\end{itemize}
\end{itemize}

\section{Experiência Execução das Técnicas}

As técnicas de elicitação foram em suma, essenciais para o levantamentos dos requisitos em ambos os  níveis do processo, foram identificadas dificuldades em utilizar algumas técnicas em alguns níveis e xetrema facilidade em utilizar as técnicas em outros níveis. Como por exemplo:


\begin{enumerate}
	\item \textbf{Workshop:}
		\begin{enumerate}
			\item Foi a primeira técnica aplicada, utilizada principalmente nas reuniões no nível de portifólio. 
	
			\item Considerada pelo time como a técnica com maior dificuldade de ser implementada, devido as regras que deviam ser seguidas, como por exemplo, a regra de que apenas um participante pode falar por vez. Durante as reuniões sempre ocorriam atropelos enquanto um dos participantes estava expondo sua ideia. 
	
			\item As interrupções também podem ser consideradas um ponto positivo, levando em conta que quando se interrompia um participante, sempre iniciava-se um depate formal para chegarem a um entendimento em comum. 
		\end{enumerate}
	
	\item Brainstorm:
		\begin{enumerate}
			\item A técnica mais utilizada durante o processo, presente em todos os níveis. Com o brainstorm o link da ideias ficavam muito mais claras.
	
			\item No nível de portifólio, era importante para ilustrar os requisitos levantados durante os debates, e como eles se ligavam e poderiam ser desmembrados.
	
			\item No nível de programa e time, era importante para ajudar a identificar a rastreabilidade dos requisitos com o nível de portifólio, ajudando assim a elicitar sempre requisitos com mais consitência, diminuindo consideravelmente número de requisitos inconsistentes que precisavam ser corrigidos.
	Uma ótima representação de uma visão geral dos requisitos do projeto.
		\end{enumerate}
	
	\item Entrevista:
		\begin{enumerate}
			\item Técnica principalmente utilizada no nível de programa, para levantar requisitos não funcionais e features. 
			
			\item Foram aplicadas entrevistas  guiadas por questionários  com as principais dúvidas do time de requisito, tentando extrair o máximo de informações possíveis do ambiente de trabalho, restrições, regras, problema, quem excultava o trabalho. Foi utilizada a técnica 5W2H em algumas entrevistas. %(http://cms-empreenda.s3.amazonaws.com/empreenda/files_static/arquivos/2014/07/01/5W2H.pdf)
			
			\item Através das entrevistas também se obteve uma visão do valor de negócio de cada requisito, ajudando na priorização de requisitos e planejamento de iterações.
		\end{enumerate}
\end{enumerate}