\chapter[Experiências de Execução]{Experiências de Execução}

\section{Experiência de Execução do Trabalho}

O trabalho desenvolvido na disciplina de requisitos 
Uma questão fundamental na Engenharia de Requisitos é como encontrar as reais necessidades do usuário para a futura implementação do software, em muitos projetos de implementação de software têm falhado por problemas de elicitação dos requisitos do software, ou seja, os requisitos obtidos muitas vezes são incompletos, mal entendidos e ambíguos. 

No experiencia vivida pode-se perceber que a  identificação corretamente dos requisitos de software não é uma tarefa fácil. Foram utilizadas 3 técnicas de elicitação diferentes, sendo que cada uma delas foi repetida mais de uma vez.

Em geral, o cliente não consegue especificar claramente tudo que precisa, prever e imaginar suas necessidades.

\section{Execução da Disciplina}

A primeira grande dificuldade da equipe como um todo foi de se situar no projeto em si,
mesmo com a fundamentação teórica devidamente assimilado, foi necessário um investimento
de tempo para o entendimento do que realmente deveria ser feito.

A medida em que a equipe foi adquirindo maturidade em relação a este trabalho e sobre a
engenharia de requisitos, mais facil foi desenvolver o tema proposto. Tambem consideramos
de extrema valia o fato da propria equipe realizar a escolha da abordagem a ser desenvolvida,
explorando importantes caracteristicas que sao bastante valiosas no dia a dia de um engenheiro
de software, por exemplo a adaptabilidade e a tomada de decisao, que sao imprescindivel na
medida em que se tem que participar de diferentes projetos simultaneos ou nao.

Uma caracteristica que achamos muito interessante no curso de engenharia de software na
FGA e que foi mais uma vez utilizada na disciplina de Requisitos de Software é a integracao
entre duas disciplinas, por ser ministrado em conjunto com a disciplina de Modelagem de
Processos, o trabalho em equipe foi extremamente valorizado na execucao do projeto, ajudando
novamente no desenvolvimento de um bom engenheiro de software.

A disciplina demanda bastante tempo e trabalho, fator que muitas vezes obriga o aluno
a balancear suas materias, optando assim por realizar um maior esforco na disciplina
de Requisitos de Software.

A disciplina de uma forma geral pode ser considerada como crucial para a formacao
de um bom engenheiro de software, desde o inicio ate o fim do trabalho, o aluno pode
experimentar situacoes semelhantes as que ocorrem em ambientes no mercado de trabalho.
