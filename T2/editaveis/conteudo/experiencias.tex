\chapter[Experiências de Execução]{Experiências de Execução}

\section{Execução das técnicas}

As técnicas de elicitação foram em suma, essenciais para o levantamentos dos requisitos
em ambos os  níveis do processo, foram identificadas dificuldades em utilizar algumas
técnicas em alguns níveis e xetrema facilidade em utilizar as técnicas em outros níveis.
Como por exemplo:

\subsection{Workshop}
\begin{itemize}
  \item Foi a primeira técnica aplicada, utilizada principalmente nas reuniões no nível de portifólio.
  \item Considerada pelo time como a técnica com maior dificuldade de ser implementada, devido as regras que deviam ser seguidas, como por exemplo, a regra de que apenas um participante pode falar por vez. Durante as reuniões sempre ocorriam atropelos enquanto um dos participantes estava expondo sua ideia.
  \item As interrupções também podem ser consideradas um ponto positivo, levando em conta que quando se interrompia um participante, sempre iniciava-se um depate formal para chegarem a um entendimento em comum.
\end{itemize}

\subsection{Brainstorm}
\begin{itemize}
\item A técnica mais utilizada durante o processo, presente em todos os níveis. Com o brainstorm o link da ideias ficavam muito mais claras.
\item No nível de portifólio, era importante para ilustrar os requisitos levantados durante os debates, e como eles se ligavam e poderiam ser desmembrados.
\item No nível de programa e time, era importante para ajudar a identificar a rastreabilidade dos requisitos com o nível de portifólio, ajudando assim a elicitar sempre requisitos com mais consitência, diminuindo consideravelmente número de requisitos inconsistentes que precisavam ser corrigidos.
\item Uma ótima representação de uma visão geral dos requisitos do projeto.
\end{itemize}

\subsection{Entrevista}
\begin{itemize}
\item Técnica principalmente utilizada no nível de programa, para levantar requisitos não funcionais e features.
\item Foram aplicadas entrevistas  guiadas por questionários  com as principais dúvidas do time de requisito, tentando extrair o máximo de informações possíveis do ambiente de trabalho, restrições, regras, problema, quem excultava o trabalho. Foi utilizada a técnica 5W2H\cite{wh} em algumas entrevistas.
\item Através das entrevistas também se obteve uma visão do valor de negócio de cada requisito, ajudando na priorização de requisitos e planejamento de iterações.
\end{itemize}


\section{Execução da Disciplina}

A primeira grande dificuldade da equipe como um todo foi de se situar no projeto em si,
mesmo com a fundamentação teórica devidamente assimilado, foi necessário um investimento
de tempo para o entendimento do que realmente deveria ser feito.

A medida em que a equipe foi adquirindo maturidade em relação a este trabalho e sobre a
engenharia de requisitos, mais facil foi desenvolver o tema proposto. Tambem consideramos
de extrema valia o fato da propria equipe realizar a escolha da abordagem a ser desenvolvida,
explorando importantes caracteristicas que sao bastante valiosas no dia a dia de um engenheiro
de software, por exemplo a adaptabilidade e a tomada de decisao, que sao imprescindivel na
medida em que se tem que participar de diferentes projetos simultaneos ou nao.

Uma caracteristica que achamos muito interessante no curso de engenharia de software na
FGA e que foi mais uma vez utilizada na disciplina de Requisitos de Software é a integracao
entre duas disciplinas, por ser ministrado em conjunto com a disciplina de Modelagem de
Processos, o trabalho em equipe foi extremamente valorizado na execucao do projeto, ajudando
novamente no desenvolvimento de um bom engenheiro de software.

A disciplina demanda bastante tempo e trabalho, fator que muitas vezes obriga o aluno
a balancear suas materias, optando assim por realizar um maior esforco na disciplina
de Requisitos de Software.

A disciplina de uma forma geral pode ser considerada como crucial para a formacao
de um bom engenheiro de software, desde o inicio ate o fim do trabalho, o aluno pode
experimentar situacoes semelhantes as que ocorrem em ambientes no mercado de trabalho.
