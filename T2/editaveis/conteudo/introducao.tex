\chapter[Introdução]{Introdução}

Uma compreensão completa do problema e a definição dos requisitos do software e sua especificação minuciosa é fundamental para o processo de desenvolvimento obter um software com alta qualidade~\cite{sommerville2007}.

Entender os requisitos de um problema está entre as tarefas mais difíceis enfrentadas por um engenheiro de software. A engenharia de requisitos ajuda a compreender melhor o problema a ser resolvido, incluindo tarefas que levam a um entendimento de qual será o impacto do software sobre o negócio, do que o cliente quer e de como os usuários finais vão interagir com o software~\cite{ariadne2001}.

Segundo \cite{pressman2011} a engenharia de requisitos começa com a concepção, tarefa que define o escopo e a natureza do problema a ser resolvido. Avançando para o levantamento que define o que é realmente necessário e depois para a elaboração, em que os requisitos básicos são refinados e modificados. Finalmente o problema é especificado de algum modo e depois revisado ou validado de modo a garantir que o seu entendimento e o entedimento do cliente sobre o problema coincidam.

Neste documento, é abordada a execução do processo de engenharia de requisitos especificado préviamente para o Processo de manutenção de software no Ministério das Comunicações, assim como o feedback desta mesma execução.
